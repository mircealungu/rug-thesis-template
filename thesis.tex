 \documentclass[oneside,a4paper,12pt]{book}
%\pagestyle{headings}
\frontmatter
\input{preamble}

% A B S T R A C T
% % % % % % % % % % % % % % % % % % % % % % % % % % % % % % % % % %
\chapter*{\centering Abstract}
\begin{quotation}
\noindent 


In less than one page, \ins{you} have to briefly provide the following information:

\begin{itemize}
	\item What is the PROBLEM you are trying to solve? Or what is the research QUESTION you are trying to answer?
	\item Why is this problem/question worth solving/asking? Who would care?
	\item How have other people in the past tried to solve/answer it?
	\item What is your NEW approach to solving/answering this problem? Or what improvement are you making on an existing solution?
	\item How do you prove that the solution you came up with is a GOOD solution?
	\item How do you demonstrate that your solution works?

\end{itemize}






\end{quotation}
\clearpage


% C O N T E N T S 
% % % % % % % % % % % % % % % % % % % % % % % % % % % % % % % % % % % % % % % %
\tableofcontents

\mainmatter
%%%%%%%%%%%%%%%%%%%%%%%%%%%%%%%%%%
%%%% NEW CHAPTER %%%%%%%%%%%%%%%%%%%%%
%%%%%%%%%%%%%%%%%%%%%%%%%%%%%%%%%%
\chapter{Introduction}
\label{cha:introduction}
Start a summary of the problem you are trying to solve and your approach.

\paragraph {Advice: Use Macros}
Probably in the introduction, you will introduce the name of your solution, and maybe other concepts that you will reuse during the thesis. Remember that LaTeX is a programming language, and you can define macros for the concepts that are frequently referred in the thesis. 

\newcommand{\tct}{The Cool Tool}

For sure the name of your tool needs to be should be defined in a macro, and then always used as such. E.g. \tct. 


\paragraph{Structure of this Thesis}
The structure of this document is the following:

\begin{itemize}
	\item In Chapter 2 (Related Work), the general research field is described together with descriptions of how others tried to solve the problem...
	\item In Chapter 3 (...) ...
	\item ...
\end{itemize}



\chapter {Related Work}
Describe the field in general and how others have tried to solve this problem.
\ml{I'll leave comments in the text when needed.}

Your goals are to:
\begin{itemize}
	\item show you are aware of current state of knowledge
(theoretical, methodological, applied) that relates to your
research topic
	\item To indicate a gap/question worthy of investigation 
\end{itemize}




\chapter{The Problem Statement}
describe in detail the problem you are trying to solve.

\chapter {The Approach}
describe your approach to solving the problem. Describe any potential weaknesses of your approach.



\chapter {The Implementation}
describe how you implemented your approach. If it is a software system give diagrams, relevant algorithms etc.

packages that you can look at for code formatting: 

- http://mirror.unl.edu/ctan/macros/latex/contrib/minted/minted.pdf
- https://en.wikibooks.org/wiki/LaTeX/Source\_Code\_Listings


\chapter {The Evaluation}
describe how you evaluated to show that your approach was successful. You may need a methods section, a results section and a conclusion section.


\chapter {Conclusion and Future Work}
Summarize your thesis again as in the introduction. 
Then you can list at a high-level the contributions of this thesis:

\paragraph{Contributions of this thesis}
Early on in the conclusions, it is a good place to list the contributions of the thesis in an explict way, so the reader is reminded of the important parts that he read. For example: 

\begin{itemize}
	\item An implementation of an automatic system that ... can teach one how to program Java in 21 days
	\item A study which evaluates the usability of the 
	\item ...
	\item ...
\end{itemize}



\section{Future Work} % (fold)
\label{sub:future_work}
A good thesis opens more questions than it answers. You now, hopefully, have a much better understading of the problem and of your solution than anybody else, and can think of the most important directionsin which the work can be continued. Think of what are the most important things that need to be done to improve the existing solution.

% subsection future_work (end)




%END Doc
%-------------------------------------------------------

\bibliography{thesis}
\bibliographystyle{plain}

\end{document}